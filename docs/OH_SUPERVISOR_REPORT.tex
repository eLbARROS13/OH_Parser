\documentclass[11pt]{article}

\usepackage[margin=1in]{geometry}
\usepackage{graphicx}
\usepackage{longtable}
\usepackage{booktabs}
\usepackage{pdflscape}
\usepackage{tcolorbox}
\usepackage{xcolor}
\usepackage{hyperref}

\definecolor{infoblue}{HTML}{1F6FEB}
\definecolor{infobg}{HTML}{EAF2FF}
\definecolor{tipgreen}{HTML}{2DA44E}
\definecolor{tipbg}{HTML}{E9F8EF}
\definecolor{warningyellow}{HTML}{D29922}
\definecolor{warningbg}{HTML}{FFF4E5}
\definecolor{lightgray}{HTML}{F6F8FA}
\definecolor{darkgray}{HTML}{57606A}

\graphicspath{{plots/hypotheses/}}

\begin{document}

\begin{center}
\Large\textbf{Occupational Health Toolkit – Statistical Report for Supervisor}\\[0.4em]
\normalsize February 1, 2026
\end{center}

\begin{tcolorbox}[colback=infobg, colframe=infoblue, title={Purpose}, fonttitle=\bfseries]
This report documents the statistical workflow, model choices, transformations, assumption checks, multiple-comparison corrections, and results for hypotheses H1–H6. It is a rigorous methods + results + interpretation briefing (not a paper) intended to justify why each model and adjustment was chosen and to summarize outputs with diagnostics.
\end{tcolorbox}

\section*{Study Design and Data Structure}

\begin{tcolorbox}[colback=tipbg, colframe=tipgreen, title={Dataset Summary}, fonttitle=\bfseries]
\begin{itemize}
\item Population: 38 office workers (Front-Office vs Back-Office)
\item Repeated measures: daily observations per subject (\texttt{subject\_id})
\item Key outcomes: EMG trapezius activity, perceived workload, sitting behavior, postural sway, and self-report sitting (OSPAQ)
\item Modeling strategy: Linear Mixed Models (LMMs) for repeated measures; OLS for subject-level validation (H4)
\item Repeated-measures implication: within-subject correlation requires random effects to avoid inflated type-I error
\end{itemize}
\end{tcolorbox}

\section*{Hypotheses and Models (Confirmatory vs Exploratory)}

All confirmatory models use ML estimation for valid likelihood ratio tests (LRT). Day effects are categorical (\texttt{C(day\_index)}), avoiding linear-trend assumptions.

\clearpage
\begin{landscape}
\footnotesize
\setlength{\LTpre}{0pt}
\setlength{\LTpost}{0pt}
\setlength{\tabcolsep}{4pt}
\renewcommand{\arraystretch}{1.2}
\setlength{\LTleft}{0pt}
\setlength{\LTright}{0pt}
\begin{longtable}{p{3.0cm}p{2.8cm}p{1.4cm}p{8.0cm}p{4.5cm}}
\textbf{Hypothesis} & \textbf{Outcome} & \textbf{Model} & \textbf{Formula (fixed effects)} & \textbf{Notes} \\
\hline
H1 (Confirmatory) & EMG p90 (\%MVC) & LMM & log(EMG p90) ~ \texttt{work\_type} + \texttt{C(day\_index)} & Log transform for skew/heteroscedasticity; EMG p90 not bounded in [0,1] \\
H2 (Confirmatory) & Workload mean & LMM & \texttt{workload\_mean} ~ \texttt{work\_type} + \texttt{C(day\_index)} & No transform \\
H3 (Confirmatory) & Sitting proportion & LMM & logit(\texttt{har\_sentado\_prop}) ~ \texttt{workload\_mean} + \texttt{work\_type} + \texttt{C(day\_index)} & Proportion -> logit \\
H4 (Confirmatory) & OSPAQ validation & OLS & logit(\texttt{har\_sentado\_prop}) ~ \texttt{ospaq\_sitting\_frac} + \texttt{work\_type} & Subject-level aggregation \\
H5 (Exploratory) & EMG p90 (\%MVC) & LMM & EMG p90 ~ \texttt{hr\_ratio\_mean\_within} + \texttt{hr\_ratio\_mean\_between} + \texttt{noise\_mean\_within} + \texttt{noise\_mean\_between} + \texttt{posture\_95\_confidence\_ellipse\_area\_within} + \texttt{posture\_95\_confidence\_ellipse\_area\_between} + \texttt{work\_type} + \texttt{C(day\_index)} & Within-between decomposition \\
H6 (Confirmatory) & Posture area ($cm^2$) & LMM & \texttt{posture\_95\_confidence\_ellipse\_area} ~ \texttt{work\_type} + \texttt{C(day\_index)} & No transform \\
\end{longtable}
\end{landscape}

\section*{Transformations and Units}

\begin{tcolorbox}[colback=lightgray, colframe=darkgray, title={Transform Summary}, fonttitle=\bfseries]
\begin{itemize}
\item EMG p90: \%MVC; can exceed 100\% -> not a strict proportion; log transform used in H1
\item Workload mean: questionnaire score; no transform
\item Sitting proportion: proportion in (0,1); logit transform used in H3 and H4
\item OSPAQ sitting: proportion in (0,1); predictor, no transform
\item Posture ellipse area: $cm^2$; no transform
\item HAR durations: seconds; used for duration-weighted sitting proportion in H4
\end{itemize}
\end{tcolorbox}

\section*{Assumption Checks and Corrections}

\begin{tcolorbox}[colback=warningbg, colframe=warningyellow, title={Diagnostics}, fonttitle=\bfseries]
\begin{itemize}
\item Normality: Q–Q plot + Shapiro-Wilk/Jarque-Bera summary
\item Homoscedasticity: residuals vs fitted + Breusch–Pagan proxy
\item Outliers: standardized residuals > 3 flagged
\item Auto-correction: if violations and outcome $\ge$ 0, apply log transform and refit
\item Bootstrap: cluster bootstrap p-values if violations persist and configured
\end{itemize}
\end{tcolorbox}

Note: H5 produced convergence warnings (boundary of parameter space), retained to avoid masking instability in the exploratory model.

\section*{Multiple Comparisons}

Confirmatory family: H1, H2, H3, H4, H6

\begin{tcolorbox}[colback=infobg, colframe=infoblue, title={Correction Strategy}, fonttitle=\bfseries]
Holm step-down procedure (FWER control) applied to the confirmatory family. H5 is exploratory and excluded from correction. Primary p-values are LRT (full vs reduced model); Wald p-values retained for sensitivity.
\end{tcolorbox}

\section*{Results: Estimates + Interpretation}

\clearpage
\begin{landscape}
\footnotesize
\setlength{\LTpre}{0pt}
\setlength{\LTpost}{0pt}
\setlength{\tabcolsep}{4pt}
\renewcommand{\arraystretch}{1.2}
\setlength{\LTleft}{0pt}
\setlength{\LTright}{0pt}
\begin{longtable}{p{2.0cm}p{1.9cm}p{2.4cm}p{4.2cm}p{1.1cm}p{1.1cm}p{1.1cm}p{7.0cm}}
\textbf{Hypothesis} & \textbf{N\_obs / N\_subjects} & \textbf{Primary term} & \textbf{Estimate (95\% CI)} & \textbf{Wald p} & \textbf{LRT p} & \textbf{Holm p} & \textbf{Interpretation} \\
\hline
H1 & 161 / 38 & \texttt{work\_type} & 0.5027 [0.1335, 0.8719] & 0.0076 & 0.0107 & 0.0537 & FO higher than BO on log scale; narrowly misses Holm threshold \\
H2 & 176 / 38 & \texttt{work\_type} & -0.0176 [-0.3930, 0.3578] & 0.9268 & 0.9268 & 1.0000 & No evidence of FO/BO workload difference \\
H3 & 168 / 38 & \texttt{workload\_mean} & -0.0505 [-0.1416, 0.0405] & 0.2768 & 0.0431 & 0.1294 & LRT suggests model-level improvement but coefficient not significant; tentative evidence \\
H4 & 38 / 38 & \texttt{ospaq\_sitting\_frac} & 0.1709 [-0.4316, 0.7734] & 0.5684 & --- & 1.0000 & No evidence of strong self-report vs objective association \\
H5 (Expl.) & 160 / 38 & \texttt{posture\_within} & 0.0041 [-0.0834, 0.0916] & 0.9266 & 0.9266 & --- & No within-day posture–EMG association; exploratory with convergence warnings \\
H6 & 180 / 38 & \texttt{work\_type} & -0.1482 [-0.2660, -0.0305] & 0.0136 & 0.0174 & 0.0698 & Suggestive FO/BO difference; not confirmatory under Holm \\
\end{longtable}
\end{landscape}

\section*{Plots and Diagnostics}

Each LMM includes a 4-panel summary: trajectories, random intercepts, Q–Q plot, residuals vs fitted.

\textbf{H1 – EMG p90}\\
\begin{center}
\includegraphics[width=0.9\textwidth]{H1\_summary.png}
\includegraphics[width=0.9\textwidth]{H1\_group\_comparison.png}
\end{center}
\textit{Diagnostics: improved normality after log transform; residual variance stabilized.}

\textbf{H2 – Workload}\\
\begin{center}
\includegraphics[width=0.9\textwidth]{H2\_summary.png}
\includegraphics[width=0.9\textwidth]{H2\_group\_comparison.png}
\end{center}
\textit{Diagnostics: approximately symmetric residuals; no strong heteroscedasticity.}

\textbf{H3 – Workload → Sitting}\\
\begin{center}
\includegraphics[width=0.9\textwidth]{H3\_summary.png}
\includegraphics[width=0.9\textwidth]{H3\_workload\_vs\_sitting.png}
\end{center}
\textit{Diagnostics: logit scale yields acceptable residual structure; mild tail deviation.}

\textbf{H4 – OSPAQ Validation}\\
\begin{center}
\includegraphics[width=0.9\textwidth]{H4\_ospaq\_vs\_objective.png}
\includegraphics[width=0.9\textwidth]{H4\_ols\_diagnostics.png}
\end{center}
\textit{Diagnostics: OLS residuals show no strong pattern; normality acceptable for N=38.}

\textbf{H5 – Physiological → EMG (Exploratory)}\\
\begin{center}
\includegraphics[width=0.9\textwidth]{H5\_summary.png}
\includegraphics[width=0.9\textwidth]{H5\_posture\_vs\_emg.png}
\end{center}
\textit{Diagnostics: convergence warnings; interpret cautiously.}

\textbf{H6 – Posture}\\
\begin{center}
\includegraphics[width=0.9\textwidth]{H6\_summary.png}
\includegraphics[width=0.9\textwidth]{H6\_group\_comparison.png}
\end{center}
\textit{Diagnostics: residuals broadly acceptable.}

\section*{Limitations and Practical Considerations}

\begin{itemize}
\item Sample size (38 subjects) limits power under strict FWER control
\item LMMs assume Gaussian residuals on the transformed scale; diagnostics support adequacy but not certainty
\item EMG p90 can exceed 100\% MVC; logit is invalid for this outcome
\item H5 is exploratory and relatively complex for the available sample size
\item H4 is cross-sectional; Bland–Altman could be added if agreement (not only correlation) is desired
\end{itemize}

\section*{Recommendations for Next Steps}

\begin{itemize}
\item Provide an FDR sensitivity analysis if a less conservative correction is desired
\item Consider power analysis for future data collections
\item Add Bland–Altman analysis for OSPAQ validation if agreement metrics are needed
\end{itemize}

\end{document}% Options for packages loaded elsewhere
\PassOptionsToPackage{unicode}{hyperref}
\PassOptionsToPackage{hyphens}{url}
%
\documentclass[
]{article}
\usepackage{amsmath,amssymb}
\usepackage{iftex}
\ifPDFTeX
  \usepackage[T1]{fontenc}
  \usepackage[utf8]{inputenc}
  \usepackage{textcomp} % provide euro and other symbols
\else % if luatex or xetex
  \usepackage{unicode-math} % this also loads fontspec
  \defaultfontfeatures{Scale=MatchLowercase}
  \defaultfontfeatures[\rmfamily]{Ligatures=TeX,Scale=1}
\fi
\usepackage{lmodern}
\ifPDFTeX\else
  % xetex/luatex font selection
\fi
% Use upquote if available, for straight quotes in verbatim environments
\IfFileExists{upquote.sty}{\usepackage{upquote}}{}
\IfFileExists{microtype.sty}{% use microtype if available
  \usepackage[]{microtype}
  \UseMicrotypeSet[protrusion]{basicmath} % disable protrusion for tt fonts
}{}
\makeatletter
\@ifundefined{KOMAClassName}{% if non-KOMA class
  \IfFileExists{parskip.sty}{%
    \usepackage{parskip}
  }{% else
    \setlength{\parindent}{0pt}
    \setlength{\parskip}{6pt plus 2pt minus 1pt}}
}{% if KOMA class
  \KOMAoptions{parskip=half}}
\makeatother
\usepackage{xcolor}
\usepackage{graphicx}
\makeatletter
\def\maxwidth{\ifdim\Gin@nat@width>\linewidth\linewidth\else\Gin@nat@width\fi}
\def\maxheight{\ifdim\Gin@nat@height>\textheight\textheight\else\Gin@nat@height\fi}
\makeatother
% Scale images if necessary, so that they will not overflow the page
% margins by default, and it is still possible to overwrite the defaults
% using explicit options in \includegraphics[width, height, ...]{}
\setkeys{Gin}{width=\maxwidth,height=\maxheight,keepaspectratio}
% Set default figure placement to htbp
\makeatletter
\def\fps@figure{htbp}
\makeatother
\setlength{\emergencystretch}{3em} % prevent overfull lines
\providecommand{\tightlist}{%
  \setlength{\itemsep}{0pt}\setlength{\parskip}{0pt}}
\setcounter{secnumdepth}{-\maxdimen} % remove section numbering
% ============================================================
% OH Stats Guide - Professional PDF Styling
% ============================================================

\usepackage{framed}
\usepackage{xcolor}
\usepackage{mdframed}
\usepackage{fontawesome5}
\usepackage{tikz}
\usetikzlibrary{shapes.geometric,arrows,positioning}
\usepackage{tcolorbox}
\tcbuselibrary{skins,breakable}
\usepackage{pdflscape}
\usepackage{enumitem}
\usepackage{booktabs}
\usepackage{colortbl}
\usepackage{array}
\usepackage{longtable}
\usepackage{fancyhdr}
\usepackage{titlesec}

% ============================================================
% COLOR PALETTE
% ============================================================
\definecolor{codebg}{RGB}{248,248,248}
\definecolor{codeframe}{RGB}{200,200,200}
\definecolor{primaryblue}{RGB}{0,102,204}
\definecolor{accentgreen}{RGB}{40,167,69}
\definecolor{warningyellow}{RGB}{255,193,7}
\definecolor{warningbg}{RGB}{255,250,230}
\definecolor{dangerred}{RGB}{220,53,69}
\definecolor{dangerbg}{RGB}{255,235,238}
\definecolor{infoblue}{RGB}{23,162,184}
\definecolor{infobg}{RGB}{232,244,248}
\definecolor{tipgreen}{RGB}{40,167,69}
\definecolor{tipbg}{RGB}{232,248,232}
\definecolor{lightgray}{RGB}{245,245,245}
\definecolor{darkgray}{RGB}{80,80,80}
\definecolor{headergray}{RGB}{60,60,60}

% ============================================================
% SECTION STYLING
% ============================================================
\titleformat{\section}
  {\normalfont\Large\bfseries\color{primaryblue}}
  {\thesection}{1em}{}
\titleformat{\subsection}
  {\normalfont\large\bfseries\color{darkgray}}
  {\thesubsection}{1em}{}
\titleformat{\subsubsection}
  {\normalfont\normalsize\bfseries\color{darkgray}}
  {\thesubsubsection}{1em}{}

% ============================================================
% CODE BLOCK STYLING
% ============================================================
\makeatletter
\@ifundefined{Shaded}{
  \newenvironment{Shaded}{
    \begin{mdframed}[
      backgroundcolor=codebg,
      linecolor=codeframe,
      linewidth=1pt,
      topline=true,
      bottomline=true,
      leftline=true,
      rightline=true,
      innerleftmargin=10pt,
      innerrightmargin=10pt,
      innertopmargin=8pt,
      innerbottommargin=8pt,
      skipabove=10pt,
      skipbelow=10pt,
      roundcorner=3pt
    ]
  }{\end{mdframed}}
}{}
\makeatother

% Reduce code font size
\AtBeginEnvironment{Highlighting}{\small}

% ============================================================
% CALLOUT BOXES
% ============================================================

% Info box (blue)
\newtcolorbox{infobox}[1][]{
  enhanced,
  breakable,
  colback=infobg,
  colframe=infoblue,
  coltitle=white,
  fonttitle=\bfseries,
  title={\faInfoCircle\ #1},
  left=8pt,
  right=8pt,
  top=6pt,
  bottom=6pt,
  boxrule=0pt,
  leftrule=4pt,
  arc=2pt
}

% Warning box (yellow)
\newtcolorbox{warningbox}[1][]{
  enhanced,
  breakable,
  colback=warningbg,
  colframe=warningyellow,
  coltitle=black,
  fonttitle=\bfseries,
  title={\faExclamationTriangle\ #1},
  left=8pt,
  right=8pt,
  top=6pt,
  bottom=6pt,
  boxrule=0pt,
  leftrule=4pt,
  arc=2pt
}

% Tip box (green)
\newtcolorbox{tipbox}[1][]{
  enhanced,
  breakable,
  colback=tipbg,
  colframe=tipgreen,
  coltitle=white,
  fonttitle=\bfseries,
  title={\faLightbulb\ #1},
  left=8pt,
  right=8pt,
  top=6pt,
  bottom=6pt,
  boxrule=0pt,
  leftrule=4pt,
  arc=2pt
}

% Danger/Important box (red)
\newtcolorbox{dangerbox}[1][]{
  enhanced,
  breakable,
  colback=dangerbg,
  colframe=dangerred,
  coltitle=white,
  fonttitle=\bfseries,
  title={\faExclamationCircle\ #1},
  left=8pt,
  right=8pt,
  top=6pt,
  bottom=6pt,
  boxrule=0pt,
  leftrule=4pt,
  arc=2pt
}

% ============================================================
% TABLE STYLING
% ============================================================
\renewcommand{\arraystretch}{1.3}
\newcolumntype{L}[1]{>{\raggedright\arraybackslash}p{#1}}
\newcolumntype{C}[1]{>{\centering\arraybackslash}p{#1}}
\newcolumntype{R}[1]{>{\raggedleft\arraybackslash}p{#1}}

% ============================================================
% HEADER/FOOTER
% ============================================================
\pagestyle{fancy}
\fancyhf{}
\fancyhead[L]{\small\textcolor{darkgray}{OH Stats Guide}}
\fancyhead[R]{\small\textcolor{darkgray}{\thepage}}
\fancyfoot[C]{\small\textcolor{darkgray}{oh\_stats v0.3.0}}
\renewcommand{\headrulewidth}{0.4pt}
\renewcommand{\footrulewidth}{0pt}

% ============================================================
% QUOTE STYLING (for blockquotes)
% ============================================================
\renewenvironment{quote}{
  \begin{mdframed}[
    backgroundcolor=lightgray,
    linecolor=primaryblue,
    linewidth=0pt,
    leftline=true,
    leftmargin=0pt,
    innerleftmargin=15pt,
    innerrightmargin=10pt,
    innertopmargin=8pt,
    innerbottommargin=8pt,
    skipabove=10pt,
    skipbelow=10pt,
    leftlinewidth=4pt
  ]
  \itshape
}{\end{mdframed}}

% ============================================================
% MISC
% ============================================================
\setlength{\parskip}{0.5em}
\setlength{\parindent}{0pt}
\ifLuaTeX
  \usepackage{selnolig}  % disable illegal ligatures
\fi
\IfFileExists{bookmark.sty}{\usepackage{bookmark}}{\usepackage{hyperref}}
\IfFileExists{xurl.sty}{\usepackage{xurl}}{} % add URL line breaks if available
\urlstyle{same}
\hypersetup{
  hidelinks,
  pdfcreator={LaTeX via pandoc}}

\author{}
\date{}

\begin{document}

\begin{center}
\Large\textbf{Occupational Health Toolkit – Statistical Report for Supervisor}\\[0.4em]
\normalsize February 1, 2026
\end{center}

\begin{tcolorbox}[colback=infobg, colframe=infoblue, title={\faIcon{info-circle} Purpose}, fonttitle=\bfseries]
This report documents the statistical workflow, model choices, transformations, assumption checks, multiple-comparison corrections, and results for hypotheses H1–H6. It is a rigorous methods + results + interpretation briefing (not a paper) intended to justify why each model and adjustment was chosen and to summarize outputs with diagnostics.
\end{tcolorbox}

\begin{center}\rule{0.5\linewidth}{0.5pt}\end{center}

\hypertarget{study-design-and-data-structure}{%
\section{Study Design and Data
Structure}\label{study-design-and-data-structure}}

\begin{tcolorbox}[colback=tipbg, colframe=tipgreen, title={\faIcon{database} Dataset Summary}, fonttitle=\bfseries]
\begin{itemize}
\item Population: 38 office workers (Front-Office vs Back-Office)
\item Repeated measures: daily observations per subject (\texttt{subject\_id})
\item Key outcomes: EMG trapezius activity, perceived workload, sitting behavior, postural sway, and self-report sitting (OSPAQ)
\item Modeling strategy: Linear Mixed Models (LMMs) for repeated measures; OLS for subject-level validation (H4)
\item Repeated-measures implication: within-subject correlation requires random effects to avoid inflated type-I error
\end{itemize}
\end{tcolorbox}

\begin{center}\rule{0.5\linewidth}{0.5pt}\end{center}

\hypertarget{hypotheses-and-models-confirmatory-vs-exploratory}{%
\section{Hypotheses and Models (Confirmatory vs
Exploratory)}\label{hypotheses-and-models-confirmatory-vs-exploratory}}

All confirmatory models use ML estimation for valid likelihood ratio
tests (LRT). Day effects are categorical (\texttt{C(day\_index)}),
avoiding linear-trend assumptions.

\begin{landscape}
\begin{longtable}{p{3.2cm}p{3.0cm}p{1.6cm}p{7.4cm}p{5.0cm}}
\textbf{Hypothesis} & \textbf{Outcome} & \textbf{Model} & \textbf{Formula (fixed effects)} & \textbf{Notes} \\
\hline
H1 (Confirmatory) & EMG p90 (\%MVC) & LMM & log(EMG p90) ~ \texttt{work\_type} + \texttt{C(day\_index)} & Log transform for skew/heteroscedasticity; EMG p90 not bounded in [0,1] \\
H2 (Confirmatory) & Workload mean & LMM & \texttt{workload\_mean} ~ \texttt{work\_type} + \texttt{C(day\_index)} & No transform \\
H3 (Confirmatory) & Sitting proportion & LMM & logit(\texttt{har\_sentado\_prop}) ~ \texttt{workload\_mean} + \texttt{work\_type} + \texttt{C(day\_index)} & Proportion -> logit \\
H4 (Confirmatory) & OSPAQ validation & OLS & logit(\texttt{har\_sentado\_prop}) ~ \texttt{ospaq\_sitting\_frac} + \texttt{work\_type} & Subject-level aggregation \\
H5 (Exploratory) & EMG p90 (\%MVC) & LMM & EMG p90 ~ \texttt{hr\_ratio\_mean\_within} + \texttt{hr\_ratio\_mean\_between} + \texttt{noise\_mean\_within} + \texttt{noise\_mean\_between} + \texttt{posture\_95\_confidence\_ellipse\_area\_within} + \texttt{posture\_95\_confidence\_ellipse\_area\_between} + \texttt{work\_type} + \texttt{C(day\_index)} & Within-between decomposition \\
H6 (Confirmatory) & Posture area ($cm^2$) & LMM & \texttt{posture\_95\_confidence\_ellipse\_area} ~ \texttt{work\_type} + \texttt{C(day\_index)} & No transform \\
\end{longtable}
\end{landscape}

\begin{center}\rule{0.5\linewidth}{0.5pt}\end{center}

\hypertarget{transformations-and-units}{%
\section{Transformations and Units}\label{transformations-and-units}}

\begin{tcolorbox}[colback=lightgray, colframe=darkgray, title={\faIcon{sliders-h} Transform Summary}, fonttitle=\bfseries]
\begin{itemize}
\item EMG p90: %MVC; can exceed 100% -> not a strict proportion; log transform used in H1
\item Workload mean: questionnaire score; no transform
\item Sitting proportion: proportion in (0,1); logit transform used in H3 and H4
\item OSPAQ sitting: proportion in (0,1); predictor, no transform
\item Posture ellipse area: $cm^2$; no transform
\item HAR durations: seconds; used for duration-weighted sitting proportion in H4
\end{itemize}
\end{tcolorbox}

\begin{center}\rule{0.5\linewidth}{0.5pt}\end{center}

\hypertarget{assumption-checks-and-corrections}{%
\section{Assumption Checks and
Corrections}\label{assumption-checks-and-corrections}}

\begin{tcolorbox}[colback=warningbg, colframe=warningyellow, title={\faIcon{exclamation-triangle} Diagnostics}, fonttitle=\bfseries]
\begin{itemize}
\item Normality: Q–Q plot + Shapiro-Wilk/Jarque-Bera summary
\item Homoscedasticity: residuals vs fitted + Breusch–Pagan proxy
\item Outliers: standardized residuals > 3 flagged
\item Auto-correction: if violations and outcome >= 0, apply log transform and refit
\item Bootstrap: cluster bootstrap p-values if violations persist and configured
\end{itemize}
\end{tcolorbox}

Note: H5 produced convergence warnings (boundary of parameter space),
retained to avoid masking instability in the exploratory model.

\begin{center}\rule{0.5\linewidth}{0.5pt}\end{center}

\hypertarget{multiple-comparisons}{%
\section{Multiple Comparisons}\label{multiple-comparisons}}

Confirmatory family: H1, H2, H3, H4, H6

\begin{tcolorbox}[colback=infobg, colframe=infoblue, title={\faIcon{check-circle} Correction Strategy}, fonttitle=\bfseries]
Holm step-down procedure (FWER control) applied to the confirmatory family. H5 is exploratory and excluded from correction. Primary p-values are LRT (full vs reduced model); Wald p-values retained for sensitivity.
\end{tcolorbox}

\begin{center}\rule{0.5\linewidth}{0.5pt}\end{center}

\hypertarget{results-estimates-interpretation}{%
\section{Results: Estimates +
Interpretation}\label{results-estimates-interpretation}}

\begin{landscape}
\begin{longtable}{p{2.4cm}p{2.2cm}p{2.8cm}p{3.4cm}p{1.3cm}p{1.3cm}p{1.3cm}p{6.6cm}}
\textbf{Hypothesis} & \textbf{N\_obs / N\_subjects} & \textbf{Primary term} & \textbf{Estimate (95\% CI)} & \textbf{Wald p} & \textbf{LRT p} & \textbf{Holm p} & \textbf{Interpretation} \\
\hline
H1 & 161 / 38 & \texttt{work\_type} & 0.5027 [0.1335, 0.8719] & 0.0076 & 0.0107 & 0.0537 & FO higher than BO on log scale; narrowly misses Holm threshold \\
H2 & 176 / 38 & \texttt{work\_type} & -0.0176 [-0.3930, 0.3578] & 0.9268 & 0.9268 & 1.0000 & No evidence of FO/BO workload difference \\
H3 & 168 / 38 & \texttt{workload\_mean} & -0.0505 [-0.1416, 0.0405] & 0.2768 & 0.0431 & 0.1294 & LRT suggests model-level improvement but coefficient not significant; tentative evidence \\
H4 & 38 / 38 & \texttt{ospaq\_sitting\_frac} & 0.1709 [-0.4316, 0.7734] & 0.5684 & — & 1.0000 & No evidence of strong self-report vs objective association \\
H5 (Expl.) & 160 / 38 & \texttt{posture\_within} & 0.0041 [-0.0834, 0.0916] & 0.9266 & 0.9266 & — & No within-day posture–EMG association; exploratory with convergence warnings \\
H6 & 180 / 38 & \texttt{work\_type} & -0.1482 [-0.2660, -0.0305] & 0.0136 & 0.0174 & 0.0698 & Suggestive FO/BO difference; not confirmatory under Holm \\
\end{longtable}
\end{landscape}

\begin{center}\rule{0.5\linewidth}{0.5pt}\end{center}

\hypertarget{plots-and-diagnostics}{%
\section{Plots and Diagnostics}\label{plots-and-diagnostics}}

Each LMM includes a 4-panel summary: trajectories, random intercepts,
Q--Q plot, residuals vs fitted.

\textbf{H1 – EMG p90}\textbackslash{}
\includegraphics{../plots/hypotheses/H1_summary.png}
\includegraphics{../plots/hypotheses/H1_group_comparison.png}
\textit{Diagnostics: improved normality after log transform; residual variance stabilized.}

\textbf{H2 – Workload}\textbackslash{}
\includegraphics{../plots/hypotheses/H2_summary.png}
\includegraphics{../plots/hypotheses/H2_group_comparison.png}
\textit{Diagnostics: approximately symmetric residuals; no strong heteroscedasticity.}

\textbf{H3 – Workload → Sitting}\textbackslash{}
\includegraphics{../plots/hypotheses/H3_summary.png}
\includegraphics{../plots/hypotheses/H3_workload_vs_sitting.png}
\textit{Diagnostics: logit scale yields acceptable residual structure; mild tail deviation.}

\textbf{H4 – OSPAQ Validation}\textbackslash{}
\includegraphics{../plots/hypotheses/H4_ospaq_vs_objective.png}
\includegraphics{../plots/hypotheses/H4_ols_diagnostics.png}
\textit{Diagnostics: OLS residuals show no strong pattern; normality acceptable for N=38.}

\textbf{H5 – Physiological → EMG (Exploratory)}\textbackslash{}
\includegraphics{../plots/hypotheses/H5_summary.png}
\includegraphics{../plots/hypotheses/H5_posture_vs_emg.png}
\textit{Diagnostics: convergence warnings; interpret cautiously.}

\textbf{H6 – Posture}\textbackslash{}
\includegraphics{../plots/hypotheses/H6_summary.png}
\includegraphics{../plots/hypotheses/H6_group_comparison.png}
\textit{Diagnostics: residuals broadly acceptable.}

\begin{center}\rule{0.5\linewidth}{0.5pt}\end{center}

\hypertarget{limitations-and-practical-considerations}{%
\section{Limitations and Practical
Considerations}\label{limitations-and-practical-considerations}}

\begin{itemize}
\item Sample size (38 subjects) limits power under strict FWER control
\item LMMs assume Gaussian residuals on the transformed scale; diagnostics support adequacy but not certainty
\item EMG p90 can exceed 100% MVC; logit is invalid for this outcome
\item H5 is exploratory and relatively complex for the available sample size
\item H4 is cross-sectional; Bland–Altman could be added if agreement (not only correlation) is desired
\end{itemize}

\begin{center}\rule{0.5\linewidth}{0.5pt}\end{center}

\hypertarget{recommendations-for-next-steps}{%
\section{Recommendations for Next
Steps}\label{recommendations-for-next-steps}}

\begin{itemize}
\item Provide an FDR sensitivity analysis if a less conservative correction is desired
\item Consider power analysis for future data collections
\item Add Bland–Altman analysis for OSPAQ validation if agreement metrics are needed
\end{itemize}\begin{center}
\Large\textbf{Occupational Health Toolkit – Statistical Report for Supervisor}\\[0.4em]
\normalsize February 1, 2026
\end{center}

\begin{tcolorbox}[colback=infobg, colframe=infoblue, title={\faIcon{info-circle} Purpose}, fonttitle=\bfseries]
This report documents the statistical workflow, model choices, transformations, assumption checks, multiple-comparison corrections, and results for hypotheses H1–H6. It is a rigorous methods + results + interpretation briefing (not a paper) intended to justify why each model and adjustment was chosen and to summarize outputs with diagnostics.
\end{tcolorbox}

\begin{center}\rule{0.5\linewidth}{0.5pt}\end{center}

\hypertarget{study-design-and-data-structure-1}{%
\section{Study Design and Data
Structure}\label{study-design-and-data-structure-1}}

\begin{tcolorbox}[colback=tipbg, colframe=tipgreen, title={\faIcon{database} Dataset Summary}, fonttitle=\bfseries]
\begin{itemize}
\item Population: 38 office workers (Front-Office vs Back-Office)
\item Repeated measures: daily observations per subject (\texttt{subject\_id})
\item Key outcomes: EMG trapezius activity, perceived workload, sitting behavior, postural sway, and self-report sitting (OSPAQ)
\item Modeling strategy: Linear Mixed Models (LMMs) for repeated measures; OLS for subject-level validation (H4)
\item Repeated-measures implication: within-subject correlation requires random effects to avoid inflated type-I error
\end{itemize}
\end{tcolorbox}

\begin{center}\rule{0.5\linewidth}{0.5pt}\end{center}

\hypertarget{hypotheses-and-models-confirmatory-vs-exploratory-1}{%
\section{Hypotheses and Models (Confirmatory vs
Exploratory)}\label{hypotheses-and-models-confirmatory-vs-exploratory-1}}

All confirmatory models use ML estimation for valid likelihood ratio
tests (LRT). Day effects are categorical (\texttt{C(day\_index)}),
avoiding linear-trend assumptions.

\begin{landscape}
\begin{longtable}{p{3.2cm}p{3.0cm}p{1.6cm}p{7.4cm}p{5.0cm}}
    extbf{Hypothesis} & \textbf{Outcome} & \textbf{Model} & \textbf{Formula (fixed effects)} & \textbf{Notes} \\
\hline
H1 (Confirmatory) & EMG p90 (%MVC) & LMM & log(EMG p90) ~ \texttt{work\_type} + \texttt{C(day\_index)} & Log transform for skew/heteroscedasticity; EMG p90 not bounded in [0,1] \\
H2 (Confirmatory) & Workload mean & LMM & \texttt{workload\_mean} ~ \texttt{work\_type} + \texttt{C(day\_index)} & No transform \\
H3 (Confirmatory) & Sitting proportion & LMM & logit(\texttt{har\_sentado\_prop}) ~ \texttt{workload\_mean} + \texttt{work\_type} + \texttt{C(day\_index)} & Proportion -> logit \\
H4 (Confirmatory) & OSPAQ validation & OLS & logit(\texttt{har\_sentado\_prop}) ~ \texttt{ospaq\_sitting\_frac} + \texttt{work\_type} & Subject-level aggregation \\
H5 (Exploratory) & EMG p90 (%MVC) & LMM & EMG p90 ~ \texttt{hr\_ratio\_mean\_within} + \texttt{hr\_ratio\_mean\_between} + \texttt{noise\_mean\_within} + \texttt{noise\_mean\_between} + \texttt{posture\_95\_confidence\_ellipse\_area\_within} + \texttt{posture\_95\_confidence\_ellipse\_area\_between} + \texttt{work\_type} + \texttt{C(day\_index)} & Within-between decomposition \\
H6 (Confirmatory) & Posture area ($cm^2$) & LMM & \texttt{posture\_95\_confidence\_ellipse\_area} ~ \texttt{work\_type} + \texttt{C(day\_index)} & No transform \\
\end{longtable}
\end{landscape}

\begin{center}\rule{0.5\linewidth}{0.5pt}\end{center}

\hypertarget{transformations-and-units-1}{%
\section{Transformations and Units}\label{transformations-and-units-1}}

\begin{tcolorbox}[colback=lightgray, colframe=darkgray, title={\faIcon{sliders-h} Transform Summary}, fonttitle=\bfseries]
\begin{itemize}
\item EMG p90: %MVC; can exceed 100% -> not a strict proportion; log transform used in H1
\item Workload mean: questionnaire score; no transform
\item Sitting proportion: proportion in (0,1); logit transform used in H3 and H4
\item OSPAQ sitting: proportion in (0,1); predictor, no transform
\item Posture ellipse area: $cm^2$; no transform
\item HAR durations: seconds; used for duration-weighted sitting proportion in H4
\end{itemize}
\end{tcolorbox}

\begin{center}\rule{0.5\linewidth}{0.5pt}\end{center}

\hypertarget{assumption-checks-and-corrections-1}{%
\section{Assumption Checks and
Corrections}\label{assumption-checks-and-corrections-1}}

\begin{tcolorbox}[colback=warningbg, colframe=warningyellow, title={\faIcon{exclamation-triangle} Diagnostics}, fonttitle=\bfseries]
\begin{itemize}
\item Normality: Q–Q plot + Shapiro-Wilk/Jarque-Bera summary
\item Homoscedasticity: residuals vs fitted + Breusch–Pagan proxy
\item Outliers: standardized residuals > 3 flagged
\item Auto-correction: if violations and outcome >= 0, apply log transform and refit
\item Bootstrap: cluster bootstrap p-values if violations persist and configured
\end{itemize}
\end{tcolorbox}

Note: H5 produced convergence warnings (boundary of parameter space),
retained to avoid masking instability in the exploratory model.

\begin{center}\rule{0.5\linewidth}{0.5pt}\end{center}

\hypertarget{multiple-comparisons-1}{%
\section{Multiple Comparisons}\label{multiple-comparisons-1}}

Confirmatory family: H1, H2, H3, H4, H6

\begin{tcolorbox}[colback=infobg, colframe=infoblue, title={\faIcon{check-circle} Correction Strategy}, fonttitle=\bfseries]
Holm step-down procedure (FWER control) applied to the confirmatory family. H5 is exploratory and excluded from correction. Primary p-values are LRT (full vs reduced model); Wald p-values retained for sensitivity.
\end{tcolorbox}

\begin{center}\rule{0.5\linewidth}{0.5pt}\end{center}

\hypertarget{results-estimates-interpretation-1}{%
\section{Results: Estimates +
Interpretation}\label{results-estimates-interpretation-1}}

\begin{landscape}
\begin{longtable}{p{2.4cm}p{2.2cm}p{2.8cm}p{3.4cm}p{1.3cm}p{1.3cm}p{1.3cm}p{6.6cm}}
    extbf{Hypothesis} & \textbf{N\_obs / N\_subjects} & \textbf{Primary term} & \textbf{Estimate (95\% CI)} & \textbf{Wald p} & \textbf{LRT p} & \textbf{Holm p} & \textbf{Interpretation} \\
\hline
H1 & 161 / 38 & \texttt{work\_type} & 0.5027 [0.1335, 0.8719] & 0.0076 & 0.0107 & 0.0537 & FO higher than BO on log scale; narrowly misses Holm threshold \\
H2 & 176 / 38 & \texttt{work\_type} & -0.0176 [-0.3930, 0.3578] & 0.9268 & 0.9268 & 1.0000 & No evidence of FO/BO workload difference \\
H3 & 168 / 38 & \texttt{workload\_mean} & -0.0505 [-0.1416, 0.0405] & 0.2768 & 0.0431 & 0.1294 & LRT suggests model-level improvement but coefficient not significant; tentative evidence \\
H4 & 38 / 38 & \texttt{ospaq\_sitting\_frac} & 0.1709 [-0.4316, 0.7734] & 0.5684 & — & 1.0000 & No evidence of strong self-report vs objective association \\
H5 (Expl.) & 160 / 38 & \texttt{posture\_within} & 0.0041 [-0.0834, 0.0916] & 0.9266 & 0.9266 & — & No within-day posture–EMG association; exploratory with convergence warnings \\
H6 & 180 / 38 & \texttt{work\_type} & -0.1482 [-0.2660, -0.0305] & 0.0136 & 0.0174 & 0.0698 & Suggestive FO/BO difference; not confirmatory under Holm \\
\end{longtable}
\end{landscape}

\begin{center}\rule{0.5\linewidth}{0.5pt}\end{center}

\hypertarget{plots-and-diagnostics-1}{%
\section{Plots and Diagnostics}\label{plots-and-diagnostics-1}}

Each LMM includes a 4-panel summary: trajectories, random intercepts,
Q--Q plot, residuals vs fitted.

\begin{verbatim}
extbf{H1 – EMG p90}\\
\end{verbatim}

\includegraphics{../plots/hypotheses/H1_summary.png}
\includegraphics{../plots/hypotheses/H1_group_comparison.png}
extit\{Diagnostics: improved normality after log transform; residual
variance stabilized.\}

\begin{verbatim}
extbf{H2 – Workload}\\
\end{verbatim}

\includegraphics{../plots/hypotheses/H2_summary.png}
\includegraphics{../plots/hypotheses/H2_group_comparison.png}
extit\{Diagnostics: approximately symmetric residuals; no strong
heteroscedasticity.\}

\begin{verbatim}
extbf{H3 – Workload → Sitting}\\
\end{verbatim}

\includegraphics{../plots/hypotheses/H3_summary.png}
\includegraphics{../plots/hypotheses/H3_workload_vs_sitting.png}
extit\{Diagnostics: logit scale yields acceptable residual structure;
mild tail deviation.\}

\begin{verbatim}
extbf{H4 – OSPAQ Validation}\\
\end{verbatim}

\includegraphics{../plots/hypotheses/H4_ospaq_vs_objective.png}
\includegraphics{../plots/hypotheses/H4_ols_diagnostics.png}
extit\{Diagnostics: OLS residuals show no strong pattern; normality
acceptable for N=38.\}

\begin{verbatim}
extbf{H5 – Physiological → EMG (Exploratory)}\\
\end{verbatim}

\includegraphics{../plots/hypotheses/H5_summary.png}
\includegraphics{../plots/hypotheses/H5_posture_vs_emg.png}
extit\{Diagnostics: convergence warnings; interpret cautiously.\}

\begin{verbatim}
extbf{H6 – Posture}\\
\end{verbatim}

\includegraphics{../plots/hypotheses/H6_summary.png}
\includegraphics{../plots/hypotheses/H6_group_comparison.png}
extit\{Diagnostics: residuals broadly acceptable.\}

\begin{center}\rule{0.5\linewidth}{0.5pt}\end{center}

\hypertarget{limitations-and-practical-considerations-1}{%
\section{Limitations and Practical
Considerations}\label{limitations-and-practical-considerations-1}}

\begin{itemize}
\item Sample size (38 subjects) limits power under strict FWER control
\item LMMs assume Gaussian residuals on the transformed scale; diagnostics support adequacy but not certainty
\item EMG p90 can exceed 100% MVC; logit is invalid for this outcome
\item H5 is exploratory and relatively complex for the available sample size
\item H4 is cross-sectional; Bland–Altman could be added if agreement (not only correlation) is desired
\end{itemize}

\begin{center}\rule{0.5\linewidth}{0.5pt}\end{center}

\hypertarget{recommendations-for-next-steps-1}{%
\section{Recommendations for Next
Steps}\label{recommendations-for-next-steps-1}}

\begin{itemize}
\item Provide an FDR sensitivity analysis if a less conservative correction is desired
\item Consider power analysis for future data collections
\item Add Bland–Altman analysis for OSPAQ validation if agreement metrics are needed
\end{itemize}

\end{document}
